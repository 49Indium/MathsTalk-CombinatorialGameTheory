\documentclass{beamer}
\usetheme{default}
\usepackage{amsmath, amsfonts, amssymb, graphics}
\usepackage{hyperref}

\newcommand{\nextslide}{$\dagger$}
\newcommand{\nextslides}{$\ddagger$}
\newcommand{\fuzzy}{\parallel}
\newcommand{\nfuzzy}{\nparallel}
\renewcommand{\star}{\mathord{*}}
\newcommand{\up}{\mathord{\uparrow}}
\newcommand{\doubleup}{\mathord{\Uparrow}}
\newcommand{\down}{\mathord{\downarrow}}
\newcommand{\doubledown}{\mathord{\Downarrow}}

\usepackage{xcolor}
\definecolor{HRed}{RGB}{204,51,51}

\usepackage{tikzit}
\input{hackenbush.tikzstyles}
\usetikzlibrary{decorations.pathreplacing}
\usetikzlibrary{arrows}
\usetikzlibrary{shapes.symbols}

\tikzset{
    invisible/.style={opacity=0},
    visible on/.style={alt={#1{}{invisible}}},
    alt/.code args={<#1>#2#3}{%
        \alt<#1>{\pgfkeysalso{#2}}{\pgfkeysalso{#3}} % \pgfkeysalso doesn't change the path
    },
}



\newcommand{\cut}[1]{#1}

\title{Combinatorial Game Theory\\\& The Fuzzy Consequences}
\author{Isaac Beh}
\date{}

\begin{document}
\begin{frame}
    \maketitle
\end{frame}

\begin{frame}{Introduction to Hackenbush}
    \centering
    \tikzfig{HackenbushPositions/InitialBush}
\end{frame}
\begin{frame}[noframenumbering]{Introduction to Hackenbush}
    \centering
    \onslide<12>{\textcolor{HRed}{Red Wins\\}}
    \tikzfig{HackenbushPositions/InitialSpread}
\end{frame}

\begin{frame}{Basic Strategy \& Intuition}
    \centering
    \tikzfig{HackenbushPositions/Example1}
\end{frame}

\begin{frame}{The Integers}
    \centering
    \tikzfig{HackenbushPositions/Integers}\\
    \bigskip
    {\huge$\alt<1>{1}{\alt<2>{2}{\alt<3>{5}{\alt<4>{4}{3}}}}$}\\
    \medskip
    \visible<6>{$3>0$ so Blue wins}
\end{frame}

\begin{frame}{A Non-Integer Value}
\centering
    \tikzfig{HackenbushPositions/Half}\\
    \bigskip
    {\huge$\alt<1>{a}{\alt<2>{-1+a}{-1+a+a}}$}\\
    \medskip
    \alt<1>{Blue wins, so $a>0$}{\alt<2>{Red wins, so $-1+a<0$ and $a<1$}{Second player wins, so $-1+a+a=0$ and $a=\frac{1}{2}$}}
\end{frame}

\begin{frame}{Tweedle-Dum Tweedle-Dee}
    \centering
    \tikzfig{HackenbushPositions/NegativeHalf}
\end{frame}
\begin{frame}[noframenumbering]{Tweedle-Dum Tweedle-Dee}
    \centering
    \only<1-7>{
        \onslide<7>{Second Player Wins\\}
        \tikzfig{HackenbushPositions/Flowers}
    }
    \only<8>{\Huge$$-x + x = 0$$}
\end{frame}

\begin{frame}{The Reals}
    \centering
    \tikzfig{HackenbushPositions/Values}
\end{frame}
\begin{frame}[noframenumbering]{The Reals}
\centering
    \tikzfig{HackenbushPositions/Decimal1} \quad {$\frac{5}{16}$}\\
    \bigskip
    \pause
    \tikzfig{HackenbushPositions/Decimal2} \quad {$2+\frac{1}{4} = \frac{9}{4}$}\\
    \bigskip
    \pause
    \tikzfig{HackenbushPositions/Decimal3} \quad {$\frac{1}{3}$}
\end{frame}
\begin{frame}[noframenumbering]{The Reals}
    \centering
    \scalebox{0.8}{\tikzfig{HackenbushPositions/RationalNumberLine}}
\end{frame}

\begin{frame}{Raising the Temperature \& Getting Fuzzy}
    \centering
    \tikzfig{HackenbushPositions/GreenEdge}\\
    \bigskip
    {\huge$\alt<1>{c}{\alt<2>{-1+c}{\alt<3>{-\frac{1}{2^n}+c}{\alt<4>{c+1}{\alt<5>{c+\frac{1}{2^n}}{\alt<6>{c}{\star}}}}}}$}\\
    \medskip
    \alt<1>{}{\alt<2>{Red wins, so $-1+c<0$ and $c<1$}{\alt<3>{Red wins, so $-\frac{1}{2^n}+c<0$ and $a<\frac{1}{2}$}{\alt<4>{Blue wins, so $c+1>0$ and $-1<c$}{\alt<5>{Blue wins, so $c+\frac{1}{2^n}>0$ and $-\frac{1}{2^n}<c$}{\alt<6>{$-\frac{1}{2^n}<c<\frac{1}{2^n}$ for all $n\in\mathbb{N}$}{$\star\fuzzy0$}}}}}}
    \only<7>{}
\end{frame}
\begin{frame}[noframenumbering]{Raising the Temperature \& Getting Fuzzy}
    \ctikzfig{HackenbushPositions/FuzzyStarZoomOutNumberline}
    \medspace\pause
    \begin{table}[]
        \begin{tabular}{@{}l|ll@{}}
            $A<B$       & $B-A>0$       & or Blue always wins $B-A$       \\
            $A=B$       & $B-A=0$       & or the second player wins $B-A$ \\
            $A>B$       & $B-A<0$       & or Red always wins $B-A$        \\
            $A\fuzzy B$ & $B-A\fuzzy 0$ & or the first player wins $B-A$
        \end{tabular}
    \end{table}
\end{frame}

\begin{frame}{Deeper into the Fuzz}
    \centering
    \tikzfig{HackenbushPositions/Upstar}\\
    \bigskip
    {\huge$\alt<1>{d}{\alt<2>{d+\frac{1}{2^n}}{\alt<3>{-\frac{1}{2^n}+d}{d}}}$}\\
    \medskip
    \alt<1>{}{\alt<2>{Blue wins, so $d+\frac{1}{2^n}>0$ and $-\frac{1}{2^n}<d$}{\alt<3>{Red wins, so $-\frac{1}{2^n}+d<0$ and $d<\frac{1}{2^n}$}{\alt<4>{$-\frac{1}{2^n}<d<\frac{1}{2^n}$ for all $n\in\mathbb{N}$}{$d\fuzzy 0$}}}}
    \only<5>{}
\end{frame}
\begin{frame}[noframenumbering]{Deeper into the Fuzz}
    \centering
    \tikzfig{HackenbushPositions/TwoUpstars}\\
    \bigskip
    {\huge$d+d$}\\
    \medskip
    Blue wins, so $d+d>0$ and thus $d\neq 0$.
\end{frame}

\begin{frame}{The Ups and Downs}
    \centering
    \tikzfig{HackenbushPositions/Up}\\
    \bigskip
    {\huge$\alt<1>{f}{\alt<2>{-\frac{1}{2^n}+f}{\alt<3>{f}{\up}}}$}\\
    \medskip
    \alt<1>{Blue wins, so $f>0$}{\alt<2>{Red wins, so $-\frac{1}{2^n}+f<0$ and $f<\frac{1}{2^n}$}{\alt<3>{$0<f<\frac{1}{2^n}$ for all $n\in\mathbb{N}$}{$0<\up<\frac{1}{2^n}$ for all $n\in\mathbb{N}$}}}
    \only<5>{}
\end{frame}
\begin{frame}[noframenumbering]{The Ups and Downs}
    \centering
    \begin{itemize}
        \setlength{\itemsep}{20pt}
        \item $\down:=-\up$
        \item $\doubleup:=\up+\up$ with $\doubleup - \up = \up > 0$ so $\doubleup > \up$
        \item $\doubledown:=\down+\down$
        \item $n.\up:=\overbrace{\up+\up+\cdots+\up}^{n\text{ times}}$
        \item $n.\down:=\overbrace{\down+\down+\cdots+\down}^{n\text{ times}}.$
    \end{itemize}
    \pause
    \ctikzfig{HackenbushPositions/Numberline}
\end{frame}

\begin{frame}{More Fuzziness with $\up\star$}
    \centering
    \begin{itemize}
        \newcommand{\scalefactor}{0.4}
        \setlength{\itemsep}{20pt}
        \item $\up\star:=\up+\star = \up-\star = \scalebox{\scalefactor}{\tikzfig{HackenbushPositions/UpSmall}} - \scalebox{\scalefactor}{\tikzfig{HackenbushPositions/StarSmall}} = \scalebox{\scalefactor}{\tikzfig{HackenbushPositions/UpstarSmall}}$
        \item $\down\star:=\down+\star$
        \item $n.\up\star:=n.\up+\star$
        \item $n.\down\star:=n.\down+\star$
    \end{itemize}
\end{frame}
\begin{frame}[noframenumbering]{More Fuzziness with $\up\star$}
    \centering
    \begin{itemize}
        \newcommand{\scalefactor}{0.4}
        \setlength{\itemsep}{20pt}
        \item $\up\star - \up = \star \fuzzy 0$, so $\up\star \fuzzy \up$
        \item $\up\star - \down = \up + \star + \up = \doubleup\star > 0$, so $\up\star>\down$
        \item $\up\star - \doubleup = \star - \up = \down\star \fuzzy 0$, so $\up\star\fuzzy \doubleup$
        \item $\up\star - 3.\up = \star - \doubleup = \doubledown\star < 0$, so $\up\star< 3.\up$
    \end{itemize}
    \pause
    \ctikzfig{HackenbushPositions/FuzzyNumberline}
    \pause
    \ctikzfig{HackenbushPositions/FuzzyStarNumberline}
    \pause
    \ctikzfig{HackenbushPositions/FuzzyDownStarNumberline}
\end{frame}

\begin{frame}{Other Strange Creatures}
    \begin{itemize}
        \item Other games must be:
        \begin{itemize}
            \item 2-player
            \item Deterministic
            \item No hidden information
            \item Finite
        \end{itemize}
        \newcommand{\scalefactor}{0.4}
        \item Games have braket notation like: $$\up\star = \scalebox{\scalefactor}{\tikzfig{HackenbushPositions/UpstarSmall}} = \left\langle \scalebox{\scalefactor}{\tikzfig{HackenbushPositions/EmptySmall}} \middle| \scalebox{\scalefactor}{\tikzfig{HackenbushPositions/StarSmall}}, \scalebox{\scalefactor}{\tikzfig{HackenbushPositions/EmptySmall}} \right\rangle = \left\langle 0 \middle| \star, 0 \right\rangle$$
        \item $\star n := \langle\star(n-1),\ldots,\star2,\star|\star(n-1),\ldots,\star2,\star\rangle$ and $\star n + \star m = \star (n\text{ XOR }m)$
        \ctikzfig{HackenbushPositions/FuzzyStarnNumberline}
    \end{itemize}
\end{frame}
\begin{frame}[noframenumbering]{Other Strange Creatures}
    \begin{itemize}
        \newcommand{\scalefactor}{0.4}
        \item \ctikzfig{HackenbushPositions/FuzzyPmNumberline}
        \item Hot games incentivise quick action. The temperature of the game is equal to the advantage of going first and is determined from its thermographs.
        \item The hottest components are the most attractive.
        \item Multiplication is distributive and commutative.
        \item Games are not closed under addition, e.g. \scalebox{\scalefactor}{\tikzfig{HackenbushPositions/UnclosedSmall}}
        \item Finite games are closed and form a field.
        \item Infinite game results rely upon the set theory axioms.
        \item ``Tiny" and ``miny" ($+_G$ and $-_G$) are the smallest possible positive and negative values.
        \item Every small valued game has an atomic weight or uppitiness, which is how many $\up$'s it is approximately equal to.
    \end{itemize}
\end{frame}

\begin{frame}{Applications}
    \begin{itemize}
        \setlength{\itemsep}{20pt}
        \item Pleasure and aesthetic beauty
        \item Go
        \item Timetabling?
    \end{itemize}
\end{frame}

\begin{frame}{Further Reading/Viewing}
    \begin{itemize}
        \setlength{\itemsep}{20pt}
        \item ``Winning Ways for Your Mathematical Plays" by  Elwyn R. Berlekamp, John H. Conway, and Richard K. Guy
        \item \href{https://www.youtube.com/watch?v=ZYj4NkeGPdM}{HACKENBUSH: a window to a new world of math} by Owen Maitzen
        \item \href{https://www.youtube.com/channel/UCF2fIXCrN_aOfluJ-8zEbeA}{Elwyn Berlekamp's channel}
        \item \url{https://hackenbush.xyz/}
    \end{itemize}
\end{frame}
\end{document}
